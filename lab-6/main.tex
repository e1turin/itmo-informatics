%___________________________ LAB-4 ______________________________
%>>>>>>>>>>>>>>>>>>>>>>>>>> ПЕРЕМЕННЫЕ >>>>>>>>>>>>>>>>>>>>>>>>>>>>>>>>>>>
%>>>>> Информация о кафедре
%\newcommand{\year}{2021 г.}  % Год устанавливается автоматически
\newcommand{\city}{Санкт-Петербург}  %  Футер, нижний колонтитул на титульном листе
\newcommand{\university}{Национальный исследовательский университет ИТМО}  % первая строка
\newcommand{\department}{Факультет программной инженерии и компьютерной техники}  % Вторая строка
\newcommand{\major}{Направление системного и прикладного программного обеспечения}  % Треьтя строка
%<<<<< Информация о кафедре

%>>>>> Назание работы
\newcommand{\lab}{Лабораторная работа}
\newcommand{\labnumber}{№ 5}                            % порядковый номер работы
\newcommand{\subject}{Информатика}                      % учебный предмет
\newcommand{\labtheme}{Верска в системе \LaTeX }        % Тема лабораторной работы
\newcommand{\variant}{№ 24 \textit{(основной)},\\[2mm]
                        \hspace{3.5cm}№ \; 2 \textit{(дополнительный)}}  % номер варианта работы

\newcommand{\student}{Тюрин Иван Николаевич}    % определение ФИО студента
\newcommand{\studygroup}{P3110}                 % определение учебной группы 
\newcommand{\teacher}{Балакшин П. В.,\\[1mm]    % ФИО лектора
                        Рудникова Т. В.}        % ФИО практика
%<<<<<<<<<<<<<<<<<<<<<<<<<< ПЕРЕМЕННЫЕ <<<<<<<<<<<<<<<<<<<<<<<<<<<<<<<<<<<


%>>>>>>>>>>>>>>>>>>>>>> ПРЕАМБУЛА >>>>>>>>>>>>>>>>>>>>>>>>>


%>>>>>>>>>>>>>>>>>> ПРЕАМБУЛА >>>>>>>>>>>>>>>>>>>>

\documentclass[12pt,final,oneside]{extreport}

%>>>>> Разметка документа
\usepackage[a4paper, mag=1000, left=0.13\textwidth, right=0.1\textwidth, top=1.3cm, bottom=1.3cm, headsep=0.7cm, footskip=1cm]{geometry} % По ГОСТу: left>=3cm, right=1cm, top=2cm, bottom=2cm,
\linespread{0.7} % межстройчный интервал по ГОСТу := 1.5
%<<<<< Разметка документа

%>>>>> babel c языковым пакетом НЕ должны быть первым импортируемым пакетом
\usepackage[utf8]{inputenc}
\usepackage[T1,T2A]{fontenc}
\usepackage[russian]{babel}
%<<<<<

%>>>>>> Для верски журнала
\usepackage{caption} % multicols не умеют в вставку table и figure поэтому описание для них отдельно, сами tabular можно завернуть в {centre}

\usepackage{blindtext} % Текст "рыба" - lorem ipsum...

\usepackage{multirow} % для создания многострочных ячеек таблицы
\usepackage{array}
\usepackage{diagbox}

%>>> верстка в 2 колонки
\usepackage{multicol} % многоколоночная верстка
\setlength{\columnsep}{.15\textwidth} % определение ширины разделителя между колонками
\usepackage{tikz} % пакет для векторной графики, чтобы рисовать красивый разделитель колонок
\usetikzlibrary{arrows.meta,decorations.pathmorphing,backgrounds,positioning,fit,petri}
\usepackage{multicolrule} % Для кастомизации разделителя колонок
\SetMCRule{                     % кастомизация разделителя колонок multicolrule
    width=2pt,
    custom-line={               % Tikz код для кастомизации линии разделителя
        \draw [                 % Рисовать
            decorate,           % декорированную (требуются спец настройки пакетов tikz (см. импорт выше)
            decoration={        % вид декорирования
                snake, % Тип - змейка (волнистая)
                amplitude=.5mm, % ширина волн
                pre length=0mm, % участок прямой линии от начала
                %segment length=0mm, % учасок волнистой линии
                post length=0mm % участок прямой линии от конца
            },
            line width=1pt,
            step=10pt
        ] 
        (TOP) to (BOT); % сверху и до низа колонки
    }, 
    extend-top=-5pt, % Вылезти за верхнюю границу колонки 
    extend-bot=-7pt % Вылезти за нижнюю границу колонки  
}
%<<< верстка в 2 колонки

%>>>...>> прочие полезные пакеты (математика, графика)
\usepackage{amsmath,amsthm,amssymb}
\usepackage{mathtext}
\usepackage{indentfirst}
\usepackage{graphicx}
\graphicspath{{/home/ivan/itmo/informatics/latex}}
\DeclareGraphicsExtensions{.pdf,.png,.jpg}
%\usepackage{bookmark}

% \usepackage[dvipsnames]{xcolor}
\usepackage{hyperref}  % Использование ссылок
\hypersetup{%  % Настройка разметки ссылок
    colorlinks=true,
    linkcolor=blue,
    filecolor=magenta,      
    urlcolor=magenta,
    %pdftitle={Overleaf Example},
    %pdfpagemode=FullScreen,
}
%<<<<<< Для верстки журнала



%>>>>> Использование листингов
\usepackage{listings} 
\usepackage{caption}
\DeclareCaptionFont{white}{\color{white}} 
\DeclareCaptionFormat{listing}{\colorbox{gray}{\parbox{\textwidth}{#1#2#3}}}

\captionsetup[lstlisting]{format=listing,labelfont=white,textfont=white} % Настройка вида описаний
\lstset{  % Настройки вида листинга
inputencoding=utf8, extendedchars=\true, keepspaces = true, % поддержка кириллицы и пробелов в комментариях
language={},            % выбор языка для подсветки (здесь это Pascal)
basicstyle=\small\sffamily, % размер и начертание шрифта для подсветки кода
numbers=left,               % где поставить нумерацию строк (слева\справа)
numberstyle=\tiny,          % размер шрифта для номеров строк
stepnumber=1,               % размер шага между двумя номерами строк
numbersep=5pt,              % как далеко отстоят номера строк от подсвечиваемого кода
backgroundcolor=\color{white}, % цвет фона подсветки - используем \usepackage{color}
showspaces=false,           % показывать или нет пробелы специальными отступами
showstringspaces=false,     % показывать илигнет пробелы в строках
showtabs=false,             % показывать или нет табуляцию в строках
frame=single,               % рисовать рамку вокруг кода
tabsize=2,                  % размер табуляции по умолчанию равен 2 пробелам
captionpos=t,               % позиция заголовка вверху [t] или внизу [b] 
breaklines=true,            % автоматически переносить строки (да\нет)
breakatwhitespace=false,    % переносить строки только если есть пробел
escapeinside={\%*}{*)}      % если нужно добавить комментарии в коде
}
%<<<<< Использование листингов

\sloppy % Решение проблем с переносами (с. 119 книга Львовского)
\emergencystretch=25pt


%>>>>>>>>>>>>>>>> КОМАНДЫ {Для соответствия ГОСТ} >>>>>>>>>>>>>>

\newcommand\Chapter[3]{%
    % Принимает 3 аргумента - название главы и дополнительный заголовок и ширина загловка (можно ничего)
    \refstepcounter{chapter}%
        %\hfill % заполнение отступом пространства до заголовка
    \chapter*{%
        \begin{minipage}{#3\textwidth} % Можно изменить ширину министраницы (заголовка)
            \flushleft % Выранивание заголовка по левому краю параграфа (заголовка)
            %\flushright % Выранивание заголовка по правому краю параграфа (заголовка)
            \begin{huge}%
                % Отключена нумерация глав в тексте:
                %:=% \textbf{\chaptername\ \arabic{chapter}\\}
                \textbf{#1}% Первый заголовок
            \end{huge}%
            \\[2mm]% Перенос сторки
            \begin{Huge}
                \textbf{#2}% Второй заголовок
            \end{Huge}
        \end{minipage}
    }%
    % Отключена нумерация для chapter в toc (table of contents), т.е. Оглавлении (Содержании):
    %:=% \addcontentsline{toc}{chapter}{\arabic{chapter}. #1}
    % Представление главы в содержании:
    \addcontentsline{toc}{chapter}{#1. #2.}%
}


\newcommand\Section[1]{
    % Принимает 1 аргумент - название секции
    \refstepcounter{section}
    \section*{%
        \raggedright
        % Отключена дополнительная нумерация chapter в section в тексте документа:
        %:=% \arabic{chapter}.\arabic{section}. #1}
        % Отключена любая нумарация section в тексте документа: (убрать \arabic{section}, оставить название секции)
        \arabic{section}. #1
    }
    
    % Отключена дополнительная нумерация chapter в section в toc (table of contents) Оглавлении (Содержании):
    %:=% \addcontentsline{toc}{section}{\arabic{chapter}.\arabic{section}. #1}
    \addcontentsline{toc}{section}{\arabic{section}. #1} 
}


\newcommand\Subsection[1]{
    % Принимает 1 аргумент - название подсекции
    \refstepcounter{subsection}
    \subsection*{%
        \raggedright%
        % Отключена дополнительная нумерация chapter в section в тексте документа (можно добавить отступ с помощью \hspace*{12pt}):
        %:=% \arabic{chapter}.\arabic{section}.\arabic{subsection}. #1}
        \arabic{section}. \arabic{subsection}. #1
    }
    % Отключена дополнительная нумерация chapter в section в Оглавлении (Содержании):
    %\addcontentsline{toc}{subsection}{\arabic{chapter}.\arabic{section}.\arabic{subsection}. #1}
    \addcontentsline{toc}{subsection}{\arabic{subsection}. #1}
}

\newcommand\Figure[4]{
    % Принимает 4 аргумента - название файла изображения, ее размер в тексте, описание, лэйбл (псевдоним в формате "fig:name") 

    \refstepcounter{figure}
    \begin{figure}[ht]
        \begin{center}
            \includegraphics[width=#2]{#1}
        \end{center}
        %\caption{#3}
        \begin{center}
            #3%
        \end{center}
        \label{fig:#4}
    \end{figure}
}

\newcommand\TableFigure[3]{ % multicols не умеют в table и figure, поэтому приходится извращаться % вставка таблицы с меткой рисунка
    %
    % Принимает 3 аргумента --- содержание таблицы(#3), ее описание(#3), лэйбл name(#2) (псевдоним в формате "tab:name") 
    %
    \begin{center}
        \refstepcounter{figure}
        \label{tab:#1}% лэйбл таблицы
         #3% Содержание таблицы
        % 
        % \captionof*{figure}{\flushleft \textsc\textbf{Рис. 1.}}
        \begin{flushleft}
            \textsf{\textbf{Рис. \arabic{figure}. #2}\\[8mm]} % Описание к картинке
        \end{flushleft}
    \end{center}
}








%<<<<<<<<<<<<<<<<<<<<<<<<<<<< КОМАНДЫ <<<<<<<<<<<<<<<<<<<<<<<<<<


%<<<<<<<<<<<<<<<<<<<<<< ПРЕАМБУЛА <<<<<<<<<<<<<<<<<<<<<<<<<



%%%%%%%%%%%%%%%%%%% СОДЕРЖИМОЕ ОТЧЕТА %%%%%%%%%%%%%%%%%%%%%
%%%%%%%%%%%%%%%%                          %%%%%%%%%%%%%%%%%
%>>>>>>>>>>>>>>> ''''''''''''''''''''''' >>>>>>>>>>>>>>>>>>
\begin{document}


%>>>>>>>>>>>>>>>> ОПРЕДЕЛЕНИЕ НАЗВАНИЙ >>>>>>>>>>>>>>>>>>>>
% Переоформление некоторых стандартных названий
%\renewcommand{\chaptername}{Лабораторная работа}
\renewcommand{\chaptername}{\lab\ \labnumber} % переименование глав
\def\contentsname{Содержание} % переименование оглавления
%<<<<<<<<<<<<<<<< ОПРЕДЕЛЕНИЕ НАЗВАНИЙ <<<<<<<<<<<<<<<<<<<<


%>>>>>>>>>>>>>>>>> ТИТУЛЬНАЯ СТРАНИЦА >>>>>>>>>>>>>>>>>>>>>
%>>>>>>>>>>>>>>>>>>> ТИТУЛЬНЫЙ ЛИСТ >>>>>>>>>>>>>>>>>>>>>>>
\begin{titlepage}

    % Название университета
    \begin{center}
    \textsc{%
        \university\\[5mm]
        \department\\[2mm]
        \major\\
    }

    \vfill
    \vfill
    % Название работы
    \textbf{ОТЧЁТ ПО ЛАБОРАТОРНОЙ РАБОТЕ \labnumber\\[3mm]
    курса <<\subject>> \\[6mm]
    по теме: <<\labtheme>>\\[3mm]
    Вариант \variant\\[20mm]
    }
    \end{center}


\hfill
% Информация об авторе работы и проверяющем
\begin{minipage}{.5\textwidth}
    \begin{flushright}
        
            
        Выполнил студент:\\[2mm] 
        \student\\[2mm]
        группа: \studygroup\\[5mm]

        Преподаватель:\\[2mm] 
        \teacher

    \end{flushright}
\end{minipage}

\vfill

    % Нижний колонтитул первой страницы
    \begin{center}
        \city, \the\year\,г.
    \end{center}

\end{titlepage}
%<<<<<<<<<<<<<<<<<<< ТИТУЛЬНЫЙ ЛИСТ <<<<<<<<<<<<<<<<<<<<<<<

 % Разница между include и input малозаметна в этой работе
%<<<<<<<<<<<<<<<<< ТИТУЛЬНАЯ СТРАНИЦА <<<<<<<<<<<<<<<<<<<<<


%>>>>>>>>>>>>>>>>>>>>> СОДЕРЖАНИЕ >>>>>>>>>>>>>>>>>>>>>>>>>
% Содержание
\tableofcontents
%<<<<<<<<<<<<<<<<<<<<< СОДЕРЖАНИЕ <<<<<<<<<<<<<<<<<<<<<<<<<


%>>>>>>>>>>>>>>>>>>>>>> КОД РАБОТЫ >>>>>>>>>>>>>>>>>>>>>>>>
%>>>>>>>>>>>>>>>>>>>>''''''''''''''''>>>>>>>>>>>>>>>>>>>>>>

% input чтобы ссылки в содержании отображались, с include этого не выходит
%>>>>>>>>>>>>>>>>>>>>>>>>> ВЕРСТКА СТРАНИЦЫ ЖУРНАЛА >>>>>>>>>>>>>>>>>>>>
\newpage
\refstepcounter{chapter} % уведичение счетчика глав
\addcontentsline{toc}{chapter}{Верстка статьи.} % добавление в содержание пункта главы с названием "Верстка статьи"

\refstepcounter{section} % уведичение счетчика глав
\addcontentsline{toc}{section}{Cтатья.} % добавление в содержание пункта главы с названием "Верстка статьи"
\begin{multicols}{2} % Начало верстки в 2 колонки
\noindent % Отключили отступ
нашли хорошую монетку, при бросании которой в среднем в половине случаев выпадает герб, а в половине --- решка.\\
%\begin{enumerate} % Можно бы и сделать нумерованый список, но это лишнне, видно что текст изначально набирался без этого (отступы параграфов отсутствуют)
    % \item 
    \indent 1. Провербте, что после \(k\) бросаний определен лишь отрезок длины \(2^{-k}\), в котором лежит точка $A$.\\
    \indent 2. По Алешиномк способу хозяин монетки не выявлен, если часть, указанная монеткой, содержащая по определению точку $A$, содержит также одну из <<критических>> точек $^1/_3$ или $^2/_3$. Выпишите последовательности, соответствующие этим точкам. Найдите, в какой части случаев (после $k$ бросаний) мы все еще рискуем оказаться в одной из указанных точек. (Доказав, что выпадение монетки по строго определенному заеону маловероятно, мы извавимся от первого Бориного возраждения.)\\
    \indent 3. Проверьте, что Витин способ можно описать в Алешиной терминологии так. Мы определяем с помощью монетки  число $A$, но отрезок $(0,1)$ распределяем более хитро:\\
    
    Алеша:\\[-2mm]
        \[ \left(0,\ \frac 1 4\right),\ \left(\frac {12} {16},\ \frac{13} {16}\right),\ \left(\frac{60} {64},\ \frac {61} {64}\right), \ldots \]
        
    Боря:\\[-2mm]
        \[ \left(\frac 1 4,\ \frac 2 4\right),\ \left(\frac {13} {16},\ \frac{14} {16}\right),\ \left(\frac{61} {64},\ \frac {62} {64}\right), \ldots \]
            
    Витя:\\[-2mm]
        \[ \left(\frac 2 4,\ \frac 3 4\right),\ \left(\frac {14} {16},\ \frac{15} {16}\right),\ \left(\frac{62} {64},\ \frac {63} {64}\right), \ldots \]
    
    \noindent
    При таком разбиении отрезка $(0,1)$ возникает много <<критических>> точек, как Витя предлагал их распределить?\\
    \indent 4. Докажите, что доля случаев, в которых один из школьников после $2k$ бросаний монетки получает монетку (по Витиному способу), равна сумме длин тех <<его>> частей отрезка $(0,1)$, в которых может оказаться точка $A$ почле $2k$ бросаний.\\
    \indent 5. Докажите, что Витин способ выбора владельца монетки справедлив.\\
    \indent 6. Используя задачу 1, обобщите задачу 4 для Алешиного способа в предположении, что после $2k$ бросаний владелец монетки определен.\\
    \indent 7. Докажите, что Алешин способ справедлив.\\
    \indent 8. Используя задачу 4, найдите долю случаев, в которых после $2k$ бросаний владелец монетки не определен
    \indent a) по Алешиному способу;
    \indent б) по Витиному способу.
    \indent Какой из способов рациональнее? Ответив на этот вопрос, вы ответите и на второй Борино возражение.
% \end{enumerate}


%%%%%%%%%%%%%%%%%%%--ВТОРАЯ КОЛОНКА--%%%%%%%%%%%%%%%%%%%%%%%%%%%%%%%
%%%%%%%%%%%%%%%""""""""""""""""""""""""""%%%%%%%%%%%%%%%%%%%%%%%%%%%

\begin{center}
    \textsf{
        \textbf{
            \begin{huge}
                ДВЕ ИГРЫ%
            \end{huge}
            \\[2mm]%
            Игра в пешки%
        }
    }
\end{center}

\indent Имеется прямоугольная клетчатая доска размером $m\times n$ и пешки: $m$ белых и $m$ черных. В начальной позиции пешки расположены на доске так, что в каждой вертикали имеется ровно одна белая и одна черная пешка, причем белая пешка должна быть расположена выше черной пешки (рис. \ref{tab:first}).
    
\TableFigure{first}{}{ % при помощи своей кастомной команды вставляю таблицу
                        %!!! ---- обертка table  НЕ РАБОТАЕТ В  multicolumn ---- использовать tabular без обертки !!!
    \begin{tabular}{p{2mm}|*{13}{p{0mm}|}}
        \multicolumn{14}{ c }{$m$}\\[1mm]\cline{2-14}
        &$\hspace{-6px}\circ$&&&&&$\hspace{-6px}\circ$&&&$\hspace{-6px}\circ$&$\hspace{-6px}\circ$&$\hspace{-6px}\circ$&&\\\cline{2-14}
        &&&$\hspace{-6px}\circ$&$\hspace{-6px}\circ$&&&$\hspace{-6px}\circ$&$\hspace{-6px}\circ$&&&&&$\hspace{-6px}\circ$\\\cline{2-14}
        &&$\hspace{-6px}\circ$&&&&&&&&&&&\\\cline{2-14}
        &&&&&$\hspace{-6px}\circ$&&&&&&&&\\\cline{2-14}
        \multirow[t]{14}{=}{$n$}&&&&&$\hspace{-6px}\bullet$&&&&&&&&\\\cline{2-14}
        &&&$\hspace{-6px}\bullet$&$\hspace{-6px}\bullet$&&$\hspace{-6px}\bullet$&&&&&&&\\\cline{2-14}
        &&&&&&&$\hspace{-6px}\bullet$&&&$\hspace{-6px}\bullet$&&&\\\cline{2-14}
        &$\hspace{-6px}\bullet$&&&&&&&&&&$\hspace{-6px}\bullet$&&$\hspace{-6px}\bullet$\\\cline{2-14}
        &&$\hspace{-6px}\bullet$&&&&&&$\hspace{-6px}\bullet$&$\hspace{-6px}\bullet$&&&$\hspace{-6px}\bullet$&\\\cline{2-14}
    \end{tabular}\\[10mm]
}

\indent Играют двое. За ход разрешается одну любую (свою) пешку передвинуть по вертикали на любое (ненулевое) число клеток доски вперед или назад. Перескакивать через пешки противника или занимать клетки, в которых они расположены, не разрешается. Выигрывает тот, кто запер своего соперника, не оставив ему возможности произвести очередной ход. (На рисунке \ref{tab:second} изображен проигрыш черных.) И, наконец, последнее правило: белые начинают.
\indent Определите, при каких начальных позициях выигрывают белые вне зависимости от стратегии черрных, а при каких --- черные.\\[2mm]
\TableFigure{second}{}{ % при помощи своей кастомной команды вставляю таблицу
                        %!!! ---- обертка table  НЕ РАБОТАЕТ В  multicolumn ---- использовать tabular без обертки !!!
    \begin{tabular}{p{2mm}|*{13}{p{0mm}|}}
        \multicolumn{14}{ c }{$m$}\\[1mm]\cline{2-14}
        &&&&&&&&&&&&&\\\cline{2-14}
        &&&&&&&&&&&&&\\\cline{2-14}
        &&&&&&&&&&&&&\\\cline{2-14}
        &&&&&&&&&&&&&\\\cline{2-14}
        \multirow[t]{14}{=}{$n$}&&&&&&&&&&&&&\\\cline{2-14}
        &&&&&&&&&&&&&\\\cline{2-14}
        &&&&&&&&&&&&&\\\cline{2-14}
        &$\hspace{-6px}\circ$&$\hspace{-6px}\circ$&$\hspace{-6px}\circ$&$\hspace{-6px}\circ$&$\hspace{-6px}\circ$&$\hspace{-6px}\circ$&$\hspace{-6px}\circ$&$\hspace{-6px}\circ$&$\hspace{-6px}\circ$&$\hspace{-6px}\circ$&$\hspace{-6px}\circ$&$\hspace{-6px}\circ$&$\hspace{-6px}\circ$\\\cline{2-14}
        &$\hspace{-6px}\bullet$&$\hspace{-6px}\bullet$&$\hspace{-6px}\bullet$&$\hspace{-6px}\bullet$&$\hspace{-6px}\bullet$&$\hspace{-6px}\bullet$&$\hspace{-6px}\bullet$&$\hspace{-6px}\bullet$&$\hspace{-6px}\bullet$&$\hspace{-6px}\bullet$&$\hspace{-6px}\bullet$&$\hspace{-6px}\bullet$&$\hspace{-6px}\bullet$\\\cline{2-14}
    \end{tabular}\\[10mm]
}

\begin{center}
    \textbf{\textit{(Продолжение см. на стр. 41)}}
\end{center}

%\blindtext[8]
\end{multicols}
%<<<<<<<<<<<<<<<<<<<<<<<<< ВЕРСТКА СТРАНИЦЫ ЖУРНАЛА <<<<<<<<<<<<<<<<<<<< % остнавная работа (страница журнала)
%>>>>>>>>>>>>>> РЕФЕРНСНОЕ ИЗОБРАЖЕНИЕ СТРАНИЦЫ ЖУРНАЛА >>>>>>>>>>>>>>>>
\Figure{reference}{0.8\paperwidth}{Рис. Референс странцы журнала.}{ref}

\refstepcounter{section} % увеличение счетчика глав
\addcontentsline{toc}{section}{Рис. Референс страницы журнала.} % добавление в содержание пункта главы с названием "..
%<<<<<<<<<<<<<< РЕФЕРНСНОЕ ИЗОБРАЖЕНИЕ СТРАНИЦЫ ЖУРНАЛА <<<<<<<<<<<<<<<< % референс (изображение страницы журнала)
%>>>>>>>>>>>>>>>>>>>>>> ДОПОЛНИЕТЛЬНОЕ ЗАДАНИЕ: ПОСТРОИТЬ ТАБЛИЦЫ >>>>>>>>>>>>>>>>>
\newpage
\Chapter{Дополнительное задание}{Верстка таблиц}{1}

\Section{Доп. Таблица 1.}
\begin{table}[h]
    \begin{tabular}{ll|llll|l|}
        \cline{3-7}
        \multicolumn{2}{l|}{\multirow{2}{*}{}} & \multicolumn{4}{l|}{Values} & \multirow{2}{*}{Total} \\ \cline{3-6}
        \multicolumn{2}{l|}{} & \multicolumn{1}{l|}{A} & \multicolumn{1}{l|}{B}  & \multicolumn{1}{l|}{C}  & D  & \\ \hline
        \multicolumn{1}{|l|}{\multirow{2}{*}{Range}} & min & \multicolumn{1}{l|}{4}  & \multicolumn{1}{l|}{8} & \multicolumn{1}{l|}{15} & 16 & 43                     \\ \cline{2-7} 
        \multicolumn{1}{|l|}{} & max & \multicolumn{1}{l|}{23} & \multicolumn{1}{l|}{42} & \multicolumn{1}{l|}{25} & 34 & 124 \\ \hline
        \multicolumn{2}{|l|}{Another total} & \multicolumn{1}{l|}{27} & \multicolumn{1}{l|}{50} & \multicolumn{1}{l|}{40} & 50 & \textbf{167} \\ \hline
    \end{tabular}
    %\caption{}
    \label{tab:ext1}
\end{table}
    
\Section{Доп. Таблица 2.}
\begin{table}[h]
    \begin{tabular}{l|l|l|l|l|l}
        \diagbox{n}{k}  & \textbf{0} & 1 & \textit{2}  & 3  & 4 \\ \hline
        0               & \textbf{1} & 0 & \textit{0}  & 0  & 0 \\ \hline
        1               & \textbf{1} & 1 & \textit{0}  & 0  & 0 \\ \hline
        2               & \textbf{1} & 2 & \textit{1}  & 0  & 0 \\ \hline
        3               & \textbf{1} & 3 & \textit{3}  & 1  & 0 \\ \hline
        4               & \textbf{1} & 4 & \textit{6}  & 4  & 1 \\ \hline
        5               & \textbf{1} & 5 & \textit{10} & 10 & 5
    \end{tabular}
    %\caption{}
    \label{tab:ext2}
\end{table}

%<<<<<<<<<<<<<<<<<<<<<< ДОПОЛНИЕТЛЬНОЕ ЗАДАНИЕ: ПОСТРОИТЬ ТАБЛИЦЫ <<<<<<<<<<<<<<<<< % выполнение дополнительного задания (верстка таблиц)

%<<<<<<<<<<<<<<<<<<<<,,,,,,,,,,,,,,,,<<<<<<<<<<<<<<<<<<<<<<
%<<<<<<<<<<<<<<<<<<<<<< КОД РАБОТЫ <<<<<<<<<<<<<<<<<<<<<<<<


\end{document}
%<<<<<<<<<<<<<<< ,,,,,,,,,,,,,,,,,,,,,,, <<<<<<<<<<<<<<<<<<
%%%%%%%%%%%%%%%%%                         %%%%%%%%%%%%%%%%%
%%%%%%%%%%%%%%%%%%%% СОДЕРЖИМОЕ ОТЧЕТА %%%%%%%%%%%%%%%%%%%%
