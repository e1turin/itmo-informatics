%>>>>>>>>>>>>>>>>>>>>>>>>> ВЕРСТКА СТРАНИЦЫ ЖУРНАЛА >>>>>>>>>>>>>>>>>>>>
\newpage
\refstepcounter{chapter} % уведичение счетчика глав
\addcontentsline{toc}{chapter}{Верстка статьи.} % добавление в содержание пункта главы с названием "Верстка статьи"

\refstepcounter{section} % уведичение счетчика глав
\addcontentsline{toc}{section}{Cтатья.} % добавление в содержание пункта главы с названием "Верстка статьи"
\begin{multicols}{2} % Начало верстки в 2 колонки
\noindent % Отключили отступ
нашли хорошую монетку, при бросании которой в среднем в половине случаев выпадает герб, а в половине --- решка.\\
%\begin{enumerate} % Можно бы и сделать нумерованый список, но это лишнне, видно что текст изначально набирался без этого (отступы параграфов отсутствуют)
    % \item 
    \indent 1. Провербте, что после \(k\) бросаний определен лишь отрезок длины \(2^{-k}\), в котором лежит точка $A$.\\
    \indent 2. По Алешиномк способу хозяин монетки не выявлен, если часть, указанная монеткой, содержащая по определению точку $A$, содержит также одну из <<критических>> точек $^1/_3$ или $^2/_3$. Выпишите последовательности, соответствующие этим точкам. Найдите, в какой части случаев (после $k$ бросаний) мы все еще рискуем оказаться в одной из указанных точек. (Доказав, что выпадение монетки по строго определенному заеону маловероятно, мы извавимся от первого Бориного возраждения.)\\
    \indent 3. Проверьте, что Витин способ можно описать в Алешиной терминологии так. Мы определяем с помощью монетки  число $A$, но отрезок $(0,1)$ распределяем более хитро:\\
    
    Алеша:\\[-2mm]
        \[ \left(0,\ \frac 1 4\right),\ \left(\frac {12} {16},\ \frac{13} {16}\right),\ \left(\frac{60} {64},\ \frac {61} {64}\right), \ldots \]
        
    Боря:\\[-2mm]
        \[ \left(\frac 1 4,\ \frac 2 4\right),\ \left(\frac {13} {16},\ \frac{14} {16}\right),\ \left(\frac{61} {64},\ \frac {62} {64}\right), \ldots \]
            
    Витя:\\[-2mm]
        \[ \left(\frac 2 4,\ \frac 3 4\right),\ \left(\frac {14} {16},\ \frac{15} {16}\right),\ \left(\frac{62} {64},\ \frac {63} {64}\right), \ldots \]
    
    \noindent
    При таком разбиении отрезка $(0,1)$ возникает много <<критических>> точек, как Витя предлагал их распределить?\\
    \indent 4. Докажите, что доля случаев, в которых один из школьников после $2k$ бросаний монетки получает монетку (по Витиному способу), равна сумме длин тех <<его>> частей отрезка $(0,1)$, в которых может оказаться точка $A$ почле $2k$ бросаний.\\
    \indent 5. Докажите, что Витин способ выбора владельца монетки справедлив.\\
    \indent 6. Используя задачу 1, обобщите задачу 4 для Алешиного способа в предположении, что после $2k$ бросаний владелец монетки определен.\\
    \indent 7. Докажите, что Алешин способ справедлив.\\
    \indent 8. Используя задачу 4, найдите долю случаев, в которых после $2k$ бросаний владелец монетки не определен
    \indent a) по Алешиному способу;
    \indent б) по Витиному способу.
    \indent Какой из способов рациональнее? Ответив на этот вопрос, вы ответите и на второй Борино возражение.
% \end{enumerate}


%%%%%%%%%%%%%%%%%%%--ВТОРАЯ КОЛОНКА--%%%%%%%%%%%%%%%%%%%%%%%%%%%%%%%
%%%%%%%%%%%%%%%""""""""""""""""""""""""""%%%%%%%%%%%%%%%%%%%%%%%%%%%

\begin{center}
    \textsf{
        \textbf{
            \begin{huge}
                ДВЕ ИГРЫ%
            \end{huge}
            \\[2mm]%
            Игра в пешки%
        }
    }
\end{center}

\indent Имеется прямоугольная клетчатая доска размером $m\times n$ и пешки: $m$ белых и $m$ черных. В начальной позиции пешки расположены на доске так, что в каждой вертикали имеется ровно одна белая и одна черная пешка, причем белая пешка должна быть расположена выше черной пешки (рис. \ref{tab:first}).
    
\TableFigure{first}{}{ % при помощи своей кастомной команды вставляю таблицу
                        %!!! ---- обертка table  НЕ РАБОТАЕТ В  multicolumn ---- использовать tabular без обертки !!!
    \begin{tabular}{p{2mm}|*{13}{p{0mm}|}}
        \multicolumn{14}{ c }{$m$}\\[1mm]\cline{2-14}
        &$\hspace{-6px}\circ$&&&&&$\hspace{-6px}\circ$&&&$\hspace{-6px}\circ$&$\hspace{-6px}\circ$&$\hspace{-6px}\circ$&&\\\cline{2-14}
        &&&$\hspace{-6px}\circ$&$\hspace{-6px}\circ$&&&$\hspace{-6px}\circ$&$\hspace{-6px}\circ$&&&&&$\hspace{-6px}\circ$\\\cline{2-14}
        &&$\hspace{-6px}\circ$&&&&&&&&&&&\\\cline{2-14}
        &&&&&$\hspace{-6px}\circ$&&&&&&&&\\\cline{2-14}
        \multirow[t]{14}{=}{$n$}&&&&&$\hspace{-6px}\bullet$&&&&&&&&\\\cline{2-14}
        &&&$\hspace{-6px}\bullet$&$\hspace{-6px}\bullet$&&$\hspace{-6px}\bullet$&&&&&&&\\\cline{2-14}
        &&&&&&&$\hspace{-6px}\bullet$&&&$\hspace{-6px}\bullet$&&&\\\cline{2-14}
        &$\hspace{-6px}\bullet$&&&&&&&&&&$\hspace{-6px}\bullet$&&$\hspace{-6px}\bullet$\\\cline{2-14}
        &&$\hspace{-6px}\bullet$&&&&&&$\hspace{-6px}\bullet$&$\hspace{-6px}\bullet$&&&$\hspace{-6px}\bullet$&\\\cline{2-14}
    \end{tabular}\\[10mm]
}

\indent Играют двое. За ход разрешается одну любую (свою) пешку передвинуть по вертикали на любое (ненулевое) число клеток доски вперед или назад. Перескакивать через пешки противника или занимать клетки, в которых они расположены, не разрешается. Выигрывает тот, кто запер своего соперника, не оставив ему возможности произвести очередной ход. (На рисунке \ref{tab:second} изображен проигрыш черных.) И, наконец, последнее правило: белые начинают.
\indent Определите, при каких начальных позициях выигрывают белые вне зависимости от стратегии черрных, а при каких --- черные.\\[2mm]
\TableFigure{second}{}{ % при помощи своей кастомной команды вставляю таблицу
                        %!!! ---- обертка table  НЕ РАБОТАЕТ В  multicolumn ---- использовать tabular без обертки !!!
    \begin{tabular}{p{2mm}|*{13}{p{0mm}|}}
        \multicolumn{14}{ c }{$m$}\\[1mm]\cline{2-14}
        &&&&&&&&&&&&&\\\cline{2-14}
        &&&&&&&&&&&&&\\\cline{2-14}
        &&&&&&&&&&&&&\\\cline{2-14}
        &&&&&&&&&&&&&\\\cline{2-14}
        \multirow[t]{14}{=}{$n$}&&&&&&&&&&&&&\\\cline{2-14}
        &&&&&&&&&&&&&\\\cline{2-14}
        &&&&&&&&&&&&&\\\cline{2-14}
        &$\hspace{-6px}\circ$&$\hspace{-6px}\circ$&$\hspace{-6px}\circ$&$\hspace{-6px}\circ$&$\hspace{-6px}\circ$&$\hspace{-6px}\circ$&$\hspace{-6px}\circ$&$\hspace{-6px}\circ$&$\hspace{-6px}\circ$&$\hspace{-6px}\circ$&$\hspace{-6px}\circ$&$\hspace{-6px}\circ$&$\hspace{-6px}\circ$\\\cline{2-14}
        &$\hspace{-6px}\bullet$&$\hspace{-6px}\bullet$&$\hspace{-6px}\bullet$&$\hspace{-6px}\bullet$&$\hspace{-6px}\bullet$&$\hspace{-6px}\bullet$&$\hspace{-6px}\bullet$&$\hspace{-6px}\bullet$&$\hspace{-6px}\bullet$&$\hspace{-6px}\bullet$&$\hspace{-6px}\bullet$&$\hspace{-6px}\bullet$&$\hspace{-6px}\bullet$\\\cline{2-14}
    \end{tabular}\\[10mm]
}

\begin{center}
    \textbf{\textit{(Продолжение см. на стр. 41)}}
\end{center}

%\blindtext[8]
\end{multicols}
%<<<<<<<<<<<<<<<<<<<<<<<<< ВЕРСТКА СТРАНИЦЫ ЖУРНАЛА <<<<<<<<<<<<<<<<<<<<