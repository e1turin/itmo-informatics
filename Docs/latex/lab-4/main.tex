%%%%%%%%%%%%%%%%%%%%%%%%%%%%%%%%% LAB-5 %%%%%%%%%%%%%%%%%%%%%%%%%%%%%%%%%%
%>>>>>>>>>>>>>>>>>>>>>>>>>> ПЕРЕМЕННЫЕ >>>>>>>>>>>>>>>>>>>>>>>>>>>>>>>>>>>
\newcommand{\lab}{Лабораторная работа}
\newcommand{\labnumber}{№ 3}                     % порядковый номер работы
\newcommand{\student}{Тюрин Иван Николаевич}   % определение ФИО студента
\newcommand{\studygroup}{P3110}                % определение учебной группы 
\newcommand{\variant}{№ 335047}                % номер варианта работы
\newcommand{\subject}{Информатика}             % учебный предмет
\newcommand{\teacher}{Балакшин П. В.,\\        % ФИО лектора
                    Рудникова Т. В.}         % ФИО практика
\newcommand{\labtheme}{Регулярные выражения}    % Тема лабораторной работы
%<<<<<<<<<<<<<<<<<<<<<<<<<< ПЕРЕМЕННЫЕ <<<<<<<<<<<<<<<<<<<<<<<<<<<<<<<<<<<


%>>>>>>>>>>>>>>>>>>>>>> ПРЕАМБУЛА >>>>>>>>>>>>>>>>>>>>>>>>>

%>>>>>>>>>>>>>>>>>> ПРЕАМБУЛА >>>>>>>>>>>>>>>>>>>>

\documentclass[12pt,final,oneside]{extreport}

%>>>>> Разметка документа
\usepackage[a4paper, mag=1000, left=0.13\textwidth, right=0.1\textwidth, top=1.3cm, bottom=1.3cm, headsep=0.7cm, footskip=1cm]{geometry} % По ГОСТу: left>=3cm, right=1cm, top=2cm, bottom=2cm,
\linespread{0.7} % межстройчный интервал по ГОСТу := 1.5
%<<<<< Разметка документа

%>>>>> babel c языковым пакетом НЕ должны быть первым импортируемым пакетом
\usepackage[utf8]{inputenc}
\usepackage[T1,T2A]{fontenc}
\usepackage[russian]{babel}
%<<<<<

%>>>>>> Для верски журнала
\usepackage{caption} % multicols не умеют в вставку table и figure поэтому описание для них отдельно, сами tabular можно завернуть в {centre}

\usepackage{blindtext} % Текст "рыба" - lorem ipsum...

\usepackage{multirow} % для создания многострочных ячеек таблицы
\usepackage{array}
\usepackage{diagbox}

%>>> верстка в 2 колонки
\usepackage{multicol} % многоколоночная верстка
\setlength{\columnsep}{.15\textwidth} % определение ширины разделителя между колонками
\usepackage{tikz} % пакет для векторной графики, чтобы рисовать красивый разделитель колонок
\usetikzlibrary{arrows.meta,decorations.pathmorphing,backgrounds,positioning,fit,petri}
\usepackage{multicolrule} % Для кастомизации разделителя колонок
\SetMCRule{                     % кастомизация разделителя колонок multicolrule
    width=2pt,
    custom-line={               % Tikz код для кастомизации линии разделителя
        \draw [                 % Рисовать
            decorate,           % декорированную (требуются спец настройки пакетов tikz (см. импорт выше)
            decoration={        % вид декорирования
                snake, % Тип - змейка (волнистая)
                amplitude=.5mm, % ширина волн
                pre length=0mm, % участок прямой линии от начала
                %segment length=0mm, % учасок волнистой линии
                post length=0mm % участок прямой линии от конца
            },
            line width=1pt,
            step=10pt
        ] 
        (TOP) to (BOT); % сверху и до низа колонки
    }, 
    extend-top=-5pt, % Вылезти за верхнюю границу колонки 
    extend-bot=-7pt % Вылезти за нижнюю границу колонки  
}
%<<< верстка в 2 колонки

%>>>...>> прочие полезные пакеты (математика, графика)
\usepackage{amsmath,amsthm,amssymb}
\usepackage{mathtext}
\usepackage{indentfirst}
\usepackage{graphicx}
\graphicspath{{/home/ivan/itmo/informatics/latex}}
\DeclareGraphicsExtensions{.pdf,.png,.jpg}
%\usepackage{bookmark}

% \usepackage[dvipsnames]{xcolor}
\usepackage{hyperref}  % Использование ссылок
\hypersetup{%  % Настройка разметки ссылок
    colorlinks=true,
    linkcolor=blue,
    filecolor=magenta,      
    urlcolor=magenta,
    %pdftitle={Overleaf Example},
    %pdfpagemode=FullScreen,
}
%<<<<<< Для верстки журнала



%>>>>> Использование листингов
\usepackage{listings} 
\usepackage{caption}
\DeclareCaptionFont{white}{\color{white}} 
\DeclareCaptionFormat{listing}{\colorbox{gray}{\parbox{\textwidth}{#1#2#3}}}

\captionsetup[lstlisting]{format=listing,labelfont=white,textfont=white} % Настройка вида описаний
\lstset{  % Настройки вида листинга
inputencoding=utf8, extendedchars=\true, keepspaces = true, % поддержка кириллицы и пробелов в комментариях
language={},            % выбор языка для подсветки (здесь это Pascal)
basicstyle=\small\sffamily, % размер и начертание шрифта для подсветки кода
numbers=left,               % где поставить нумерацию строк (слева\справа)
numberstyle=\tiny,          % размер шрифта для номеров строк
stepnumber=1,               % размер шага между двумя номерами строк
numbersep=5pt,              % как далеко отстоят номера строк от подсвечиваемого кода
backgroundcolor=\color{white}, % цвет фона подсветки - используем \usepackage{color}
showspaces=false,           % показывать или нет пробелы специальными отступами
showstringspaces=false,     % показывать илигнет пробелы в строках
showtabs=false,             % показывать или нет табуляцию в строках
frame=single,               % рисовать рамку вокруг кода
tabsize=2,                  % размер табуляции по умолчанию равен 2 пробелам
captionpos=t,               % позиция заголовка вверху [t] или внизу [b] 
breaklines=true,            % автоматически переносить строки (да\нет)
breakatwhitespace=false,    % переносить строки только если есть пробел
escapeinside={\%*}{*)}      % если нужно добавить комментарии в коде
}
%<<<<< Использование листингов

\sloppy % Решение проблем с переносами (с. 119 книга Львовского)
\emergencystretch=25pt


%>>>>>>>>>>>>>>>> КОМАНДЫ {Для соответствия ГОСТ} >>>>>>>>>>>>>>

\newcommand\Chapter[3]{%
    % Принимает 3 аргумента - название главы и дополнительный заголовок и ширина загловка (можно ничего)
    \refstepcounter{chapter}%
        %\hfill % заполнение отступом пространства до заголовка
    \chapter*{%
        \begin{minipage}{#3\textwidth} % Можно изменить ширину министраницы (заголовка)
            \flushleft % Выранивание заголовка по левому краю параграфа (заголовка)
            %\flushright % Выранивание заголовка по правому краю параграфа (заголовка)
            \begin{huge}%
                % Отключена нумерация глав в тексте:
                %:=% \textbf{\chaptername\ \arabic{chapter}\\}
                \textbf{#1}% Первый заголовок
            \end{huge}%
            \\[2mm]% Перенос сторки
            \begin{Huge}
                \textbf{#2}% Второй заголовок
            \end{Huge}
        \end{minipage}
    }%
    % Отключена нумерация для chapter в toc (table of contents), т.е. Оглавлении (Содержании):
    %:=% \addcontentsline{toc}{chapter}{\arabic{chapter}. #1}
    % Представление главы в содержании:
    \addcontentsline{toc}{chapter}{#1. #2.}%
}


\newcommand\Section[1]{
    % Принимает 1 аргумент - название секции
    \refstepcounter{section}
    \section*{%
        \raggedright
        % Отключена дополнительная нумерация chapter в section в тексте документа:
        %:=% \arabic{chapter}.\arabic{section}. #1}
        % Отключена любая нумарация section в тексте документа: (убрать \arabic{section}, оставить название секции)
        \arabic{section}. #1
    }
    
    % Отключена дополнительная нумерация chapter в section в toc (table of contents) Оглавлении (Содержании):
    %:=% \addcontentsline{toc}{section}{\arabic{chapter}.\arabic{section}. #1}
    \addcontentsline{toc}{section}{\arabic{section}. #1} 
}


\newcommand\Subsection[1]{
    % Принимает 1 аргумент - название подсекции
    \refstepcounter{subsection}
    \subsection*{%
        \raggedright%
        % Отключена дополнительная нумерация chapter в section в тексте документа (можно добавить отступ с помощью \hspace*{12pt}):
        %:=% \arabic{chapter}.\arabic{section}.\arabic{subsection}. #1}
        \arabic{section}. \arabic{subsection}. #1
    }
    % Отключена дополнительная нумерация chapter в section в Оглавлении (Содержании):
    %\addcontentsline{toc}{subsection}{\arabic{chapter}.\arabic{section}.\arabic{subsection}. #1}
    \addcontentsline{toc}{subsection}{\arabic{subsection}. #1}
}

\newcommand\Figure[4]{
    % Принимает 4 аргумента - название файла изображения, ее размер в тексте, описание, лэйбл (псевдоним в формате "fig:name") 

    \refstepcounter{figure}
    \begin{figure}[ht]
        \begin{center}
            \includegraphics[width=#2]{#1}
        \end{center}
        %\caption{#3}
        \begin{center}
            #3%
        \end{center}
        \label{fig:#4}
    \end{figure}
}

\newcommand\TableFigure[3]{ % multicols не умеют в table и figure, поэтому приходится извращаться % вставка таблицы с меткой рисунка
    %
    % Принимает 3 аргумента --- содержание таблицы(#3), ее описание(#3), лэйбл name(#2) (псевдоним в формате "tab:name") 
    %
    \begin{center}
        \refstepcounter{figure}
        \label{tab:#1}% лэйбл таблицы
         #3% Содержание таблицы
        % 
        % \captionof*{figure}{\flushleft \textsc\textbf{Рис. 1.}}
        \begin{flushleft}
            \textsf{\textbf{Рис. \arabic{figure}. #2}\\[8mm]} % Описание к картинке
        \end{flushleft}
    \end{center}
}








%<<<<<<<<<<<<<<<<<<<<<<<<<<<< КОМАНДЫ <<<<<<<<<<<<<<<<<<<<<<<<<<

%<<<<<<<<<<<<<<<<<<<<<< ПРЕАМБУЛА <<<<<<<<<<<<<<<<<<<<<<<<<


%%%%%%%%%%%%%%%%%%% СОДЕРЖИМОЕ ОТЧЕТА %%%%%%%%%%%%%%%%%%%%%
%>>>>>>>>>>>>>>> ''''''''''''''''''''''' >>>>>>>>>>>>>>>>>>
\begin{document}


%>>>>>>>>>>>>>>>> ОПРЕДЕЛЕНИЕ НАЗВАНИЙ >>>>>>>>>>>>>>>>>>>>
% Переоформление некоторых стандартных названий
%\renewcommand{\chaptername}{Лабораторная работа}
\renewcommand{\chaptername}{\lab\ \labnumber} % переименование глав
\def\contentsname{Содержание} % переименование оглавления
%<<<<<<<<<<<<<<<< ОПРЕДЕЛЕНИЕ НАЗВАНИЙ <<<<<<<<<<<<<<<<<<<<


%>>>>>>>>>>>>>>>>> ТИТУЛЬНАЯ СТРАНИЦА >>>>>>>>>>>>>>>>>>>>>
%>>>>>>>>>>>>>>>>>>> ТИТУЛЬНЫЙ ЛИСТ >>>>>>>>>>>>>>>>>>>>>>>
\begin{titlepage}

    % Название университета
    \begin{center}
    \textsc{%
        \university\\[5mm]
        \department\\[2mm]
        \major\\
    }

    \vfill
    \vfill
    % Название работы
    \textbf{ОТЧЁТ ПО ЛАБОРАТОРНОЙ РАБОТЕ \labnumber\\[3mm]
    курса <<\subject>> \\[6mm]
    по теме: <<\labtheme>>\\[3mm]
    Вариант \variant\\[20mm]
    }
    \end{center}


\hfill
% Информация об авторе работы и проверяющем
\begin{minipage}{.5\textwidth}
    \begin{flushright}
        
            
        Выполнил студент:\\[2mm] 
        \student\\[2mm]
        группа: \studygroup\\[5mm]

        Преподаватель:\\[2mm] 
        \teacher

    \end{flushright}
\end{minipage}

\vfill

    % Нижний колонтитул первой страницы
    \begin{center}
        \city, \the\year\,г.
    \end{center}

\end{titlepage}
%<<<<<<<<<<<<<<<<<<< ТИТУЛЬНЫЙ ЛИСТ <<<<<<<<<<<<<<<<<<<<<<<


%<<<<<<<<<<<<<<<<< ТИТУЛЬНАЯ СТРАНИЦА <<<<<<<<<<<<<<<<<<<<<


%>>>>>>>>>>>>>>>>>>>>> СОДЕРЖАНИЕ >>>>>>>>>>>>>>>>>>>>>>>>>
% Содержание
\tableofcontents
\newpage
%<<<<<<<<<<<<<<<<<<<<< СОДЕРЖАНИЕ <<<<<<<<<<<<<<<<<<<<<<<<<


%%%%%%%%%%%%%%%%%%%%%%% КОД РАБОТЫ %%%%%%%%%%%%%%%%%%%%%%%%
%>>>>>>>>>>>>>>>>>>>'''''''''''''''''>>>>>>>>>>>>>>>>>>>>>

\Chapter{\lab\ \labnumber}{\labtheme}

\Section{Задание варианта \variant}
\begin{center}
    {, , ,}
\end{center} 
\begin{enumerate}
    \item Понять устройство страницы с расписанием для своей группы: \url{https://itmo.ru/ru/schedule/0/P3110/schedule.htm}
\item Исходя из структуры расписания конкретного дня, сформировать файл с расписанием в формате, указанном в задании в качестве исходного.
\item Обязательное задание (позволяет набрать до 65 процентов от  максимального числа баллов БаРС за данную лабораторную): написать  программу на языке Python 3.x, которая бы осуществляла парсинг и  конвертацию исходного файла в новый.
\item Нельзя использовать готовые библиотеки, в том числе регулярные  выражения в Python и библиотеки для загрузки XML-файлов.
\item Дополнительное задание задание №1 (позволяет набрать +10 процентов  от максимального числа баллов БаРС за данную лабораторную).
    \begin{itemize}
        \item[a)] Найти готовые библиотеки, осуществляющие аналогичный  парсинг и конвертацию файлов.
        \item[b)] Переписать исходный код, применив найденные библиотеки.  Регулярные выражения также нельзя использовать.
        \item[c)] Сравнить полученные результаты и объяснить их  сходство/различие.
    \end{itemize}
\item Дополнительное задание задание №2 (позволяет набрать +10 процентов  от максимального числа баллов БаРС за данную лабораторную).
    \begin{itemize}
        \item[a)] Переписать исходный код, добавив в него использование  регулярных выражений.
        \item[b)] Сравнить полученные результаты и объяснить их  сходство/различие.
    \end{itemize}
\item Дополнительное задание задание №3 (позволяет набрать +10 процентов  от максимального числа баллов БаРС за данную лабораторную).
    \begin{itemize}
        \item[a)] Используя свою исходную программу из обязательного  задания, программу из дополнительного задания №1 и  программу из дополнительного задания №2, сравнить  десятикратное время выполнения парсинга + конвертации в  цикле.
        \item[b)] Проанализировать полученные результаты и объяснить их  сходство/различие.
    \end{itemize}
\item Дополнительное задание задание №4 (позволяет набрать +5 процентов  от максимального числа баллов БаРС за данную лабораторную.
    \begin{itemize}
        \item[a)] Переписать исходную, чтобы она осуществляла парсинг и  конвертацию исходного файла в любой другой формат (кроме  JSON, YAML, XML, HTML): PROTOBUF, TSV, CSV, WML и  т.п.
        \item[b)] Проанализировать полученные результаты, объяснить  осоебнности использованного формата.
    \end{itemize}
\end{enumerate}
\begin{center}
    {` ` `}
\end{center}  


\newpage
\Section{Выполнение задания 1 и 2}
Былы просмотрена струра веб страницы с расписанием моей группы, она представляет собой текстовый документ, в формате HTML.\\

Было составленно собственное расписание пар в среду в формате YAML.
Текстовый документ размещен в личном удаленном репозитории.  Содержимое этого текстового документа можно найти по ссылке: \url{https://github.com/e1turin/itmo-informatics/blob/main/lab-4/assets/timetable.yaml}.\\


\Section{Выполнение обязательного задания}\label{main}
Была написана программа на языке программирования Python, которая парсит исходный файл с расписанием (составленный самостоятельно) в формате YAML, строит соответствующую структуру данных внутри ЯП, конвертирует эту структуру в формат XML, при этом имеется возможность вывести полученный текстовый документ в стандартный поток вывода (консоль) или записать в файл.\\

Исходный код программы размещен в личном удаленном репозитории. 
Код главного скрипта \texttt{main.py} можно найти по ссылке:\\ \url{https://github.com/e1turin/itmo-informatics/blob/main/lab-4/main.py}.\\

Для его работы были написаны мини библиотеки: \texttt{YAML.py} для работы с YAML-файлом, \texttt{XML.py} для работы с XML-файлом, \texttt{exсeptions.py} для обработки ошибок пользователя.\\
Исходный код библиотек можно найти по ссылкам: 
\begin{itemize}
    \item \texttt{YAML.py} ---  \url{https://github.com/e1turin/itmo-informatics/blob/main/lab-4/src/YAML.py}
    
    \item \texttt{XML.py} --- \url{https://github.com/e1turin/itmo-informatics/blob/main/lab-4/src/XML.py}
    
    \item \texttt{exсeptions.py} --- \url{https://github.com/e1turin/itmo-informatics/blob/main/lab-4/src/exceptions.py}
\end{itemize}


\Section{Выполнение дополнительного задания № 1}\label{main1}
Была написана программа на языке программирования Python, которая выполняет конвертацию YAML-файла в XML-файл с использованием готовых библиотек для ЯП Python.\\

Исходный код программы размещен в личном удаленном репозитории. Код можно найти по ссылке: \url{https://github.com/e1turin/itmo-informatics/blob/main/lab-4/main1.py}.\\ 

\Section{Выполнение дополнительного задания № 2}\label{main2}
Была написана программа на языке программирования Python, которая выполняет конвертацию YAML-файла в XML-файл без использования готовых библиотек для этого, но с использованием регульярных выражений.\\

Исходный код программы размещен в личном удаленном репозитории. Код можно найти по ссылке: \url{https://github.com/e1turin/itmo-informatics/blob/main/lab-4/main2.py}.\\ 
\Section{Выполнение дополнительного задания № 3}\label{main3}
Была написана программа на языке программирования Python, которая сравнивает скорость выполнения программы написанной в качестве обязательного задания (см. c.~\pageref{main}) и программы написанной в качестве дополнительного задания № 2 (см. c.~\pageref{main2})\\

Исходный код программы размещен в личном удаленном репозитории. Код можно найти по ссылке: \url{https://github.com/e1turin/itmo-informatics/blob/main/lab-4/main3.py}.\\ 


\Section{Выполнение дополнительного задания № 4}\label{main4}
Была написана программа на языке программирования Python, которая выполняет конвертацию YAML-файла в BSON-файл (Бинарный JSON) с использованием готовых библиотек для ЯП Python.\\

Исходный код программы размещен в личном удаленном репозитории. Код можно найти по ссылке: \url{https://github.com/e1turin/itmo-informatics/blob/main/lab-4/main4.py}.\\ 


\Section{Вывод}

Изучил форматы хранения данных и языки разметки документов. Научился работать с библиотеками ЯП Python для YAML, XML. Укрепил знания по работе с Python.\\
\newpage
%<<<<<<<<<<<<<<<<<<<<<< КОД РАБОТЫ <<<<<<<<<<<<<<<<<<<<<<<<


%>>>>>>>>>>>>>>>> СПИСОК ЛИТЕРАТУРЫ >>>>>>>>>>>>>>>>>>>>>>>
%\bibliographystyle{plain}
%\begin{thebibliography}{3}
    %\addcontentsline{toc}{chapter}{Список лиетратуры}
%
    %\bibitem{gutgut:1}
    %Код Хэмминга. Пример работы алгоритма. URL: \url{https://habr.com/ru/post/140611/};
%
    %\bibitem{gutgut:2}
    %Избыточное кодирование, код Хэмминга. URL: \url{https://neerc.ifmo.ru/wiki/index.php?title=%D0%98%D0%B7%D0%B1%D1%8B%D1%82%D0%BE%D1%87%D0%BD%D0%BE%D0%B5_%D0%BA%D0%BE%D0%B4%D0%B8%D1%80%D0%BE%D0%B2%D0%B0%D0%BD%D0%B8%D0%B5,_%D0%BA%D0%BE%D0%B4_%D0%A5%D1%8D%D0%BC%D0%BC%D0%B8%D0%BD%D0%B3%D0%B0}.
%\end{thebibliography}
%<<<<<<<<<<<<<<<<<<<< СПИСОК ЛИТЕРАТУРЫ <<<<<<<<<<<<<<<<<<<


\end{document}
%<<<<<<<<<<<<<<<< ,,,,,,,,,,,,,,,,,,,,,,, <<<<<<<<<<<<<<<<<
%<<<<<<<<<<<<<<<<<<< СОДЕРЖИМОЕ ОТЧЕТА <<<<<<<<<<<<<<<<<<<<
